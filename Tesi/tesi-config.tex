%**************************************************************
% file contenente le impostazioni della tesi
%**************************************************************

%**************************************************************
% Frontespizio
%**************************************************************

% Autore
\newcommand{\myName}{Marco Pozza}
\newcommand{\myTitle}{Identity Trust Fabric: sistema per la gestione dell'identità e l'accesso ai servizi basato su Blockchain}

% Tipo di tesi                   
\newcommand{\myDegree}{Tesi di laurea triennale}

% Università             
\newcommand{\myUni}{Università degli Studi di Padova}

% Facoltà       
\newcommand{\myFaculty}{Corso di Laurea in Informatica}

% Dipartimento
\newcommand{\myDepartment}{Dipartimento di Matematica "Tullio Levi-Civita"}

% Titolo del relatore
\newcommand{\profTitle}{Prof.}

% Relatore
\newcommand{\myProf}{Luigi De Giovanni}

% Luogo
\newcommand{\myLocation}{Padova}

% Anno accademico
\newcommand{\myAA}{2017-2018}

% Data discussione
\newcommand{\myTime}{Settembre 2018}


%**************************************************************
% Impostazioni di impaginazione
% see: http://wwwcdf.pd.infn.it/AppuntiLinux/a2547.htm
%**************************************************************

\setlength{\parindent}{14pt}   % larghezza rientro della prima riga
\setlength{\parskip}{0pt}   % distanza tra i paragrafi


%**************************************************************
% Impostazioni di biblatex
%**************************************************************
\bibliography{bibliografia} % database di biblatex 

\defbibheading{bibliography} {
    \cleardoublepage
    \phantomsection 
    \addcontentsline{toc}{chapter}{\bibname}
    \chapter*{\bibname\markboth{\bibname}{\bibname}}
}

\setlength\bibitemsep{1.5\itemsep} % spazio tra entry

\DeclareBibliographyCategory{opere}
\DeclareBibliographyCategory{web}

\addtocategory{opere}{womak:lean-thinking}
\addtocategory{web}{site:agile-manifesto}

\defbibheading{opere}{\section*{Riferimenti bibliografici}}
\defbibheading{web}{\section*{Siti Web consultati}}


%**************************************************************
% Impostazioni di caption
%**************************************************************
\captionsetup{
    tableposition=top,
    figureposition=bottom,
    font=small,
    format=hang,
    labelfont=bf
}

%**************************************************************
% Impostazioni di glossaries
%**************************************************************

%**************************************************************
% Acronimi
%**************************************************************
\renewcommand{\acronymname}{Acronimi e abbreviazioni}
%\newacronym[description={\glslink{ITF}{Identity Trust Fabric}}]{ITF}{Identity Trust Fabric}
\newacronym{ITF}{ITF}{Identity Trust Fabric}

%**************************************************************
% Glossario
%**************************************************************
%\renewcommand{\glossaryname}{Glossario}
%\newglossaryentry{umlg}
%{
%	name=\glslink{uml}{UML},
%	text=UML,
%	sort=uml,
%	description={in ingegneria del software \emph{UML, Unified Modeling Language} (ing. linguaggio di modellazione unificato) è un linguaggio di modellazione e specifica basato sul paradigma object-oriented. L'\emph{UML} svolge un'importantissima funzione di ``lingua franca'' nella comunità della progettazione e programmazione a oggetti. Gran parte della letteratura di settore usa tale linguaggio per descrivere soluzioni analitiche e progettuali in modo sintetico e comprensibile a un vasto pubblico}
%}
\newglossaryentry{agile}
{
	name=Agile,
	description={
		 In ingegneria del software, è un insieme di metodi di sviluppo del software emersi a partire dai primi anni 2000 e fondati su un insieme di principi comuni, direttamente o indirettamente derivati dai principi del \textit{Manifesto per lo sviluppo agile del software}\cite{manifestoAgile}.\\
		 Tale manifesto si può riassumere in quattro punti:
		 \begin{enumerate}
		 	\item le persone e le interazioni sono più importanti dei processi e degli strumenti;
		 	\item è più importante avere software funzionante che documentazione
		 	\item bisogna collaborare con i clienti oltre che rispettare il contratto;
		 	\item bisogna essere pronti a rispodere ai cambiamenti oltre che aderire alla pianificazione.
		 \end{enumerate}
		 Un esempio di metodo di sviluppo di tipo agile è il metodo \gls{Scrum}}
}
\newglossaryentry{DataManagement}
{
	name=Data Management,
	description={insime di più attività che hanno lo scopo di sviluppare, eseguire e supervisionare progetti, politiche e programmi che controllano, proteggono, trasportano e aumentano il valore di dati e di risorse informative}
}
\newglossaryentry{DataMining}
{
  name=Data Mining,
  description={Insieme di tecniche e metodologie che hanno per oggetto l'estrazione di informazioni da grandi quantità di dati attraverso
  emtodi automatici o semi-automatici}
}
\newglossaryentry{Scrum}
{
	name=Scrum,
	description={Metodo di sviluppo software che rientra fra i metodi agile. Prevede di dividere il progetto in blocchi rapidi di lavoro (\textit{sprint}) alla fine di ciascuno dei quali crea un incremento del software. Esso indica come definire i dettagli del lavoro da fare nell'immediato futuro e prevede vari meeting con caratteristiche precise per creare occasioni di ispezione e controllo del lavoro svolto \cite{scrum}
	}
}
\newglossaryentry{ProductBacklog}
{
	name= Product Backlog,
	description={Insime di requisiti per il lavoro da compiere, con priorità assegnate in base al valore di business}
}
\newglossaryentry{SprintBacklog}
{
	name= Sprint Backlog,
	description={Insieme degli elementi selezionati dal Product Backlog per lo \textit{sprint}}
}
\newglossaryentry{IAM}
{
	name=Identity and Access Management,
	description={L'Identity and Access Management è un sistema che si occupa della gestione delle informazioni riguardanti l'identità degli utenti e controlla l'accesso alle risorse informatiche. Nello specifico:
	\begin{itemize}
		\item \textbf{Identity Management} - censisce le utenze;
		\item \textbf{Access Management} - gestisce l'autenticazione e l'autorizzazione delle utenze. 
	\end{itemize}}
}
\newglossaryentry{identityWallet}{
	name=Identity Wallet,
	description={Portafogli digitali che contengono all'loro interno tutte le informazioni necessarie ad un utente per poter permettere l'autenticazione preciso servizio}
}
\newglossaryentry{monokee}{
name = Monokee,
description={Un \emph{\gls{IaaS}}\glsfirstoccur ideato e sviluppato da Athesys S.r.l per realizzare l'\gls{IAM}}
}
\newglossaryentry{IaaS}{
	name =Identity-as-a-Service,
	description ={È un servizio cloud che fornisce funzionalità di Identity and Access Management ad un sistema nel cloud e/o locale}
}
\newglossaryentry{identityProofing}{
	name=Identity Proofing,
	description={Conosciuta anche come \textit{Identity Verification} significa verificare ed autenticare l'identità di un utente prima che questo possa avere accesso ad uno o più servizi}
}
\newglossaryentry{identityBinding}{
	name=Identity Binding,
	description={Legare l'identità digitale all'individuo che la reclama attraverso vari meccanismi di riconoscimento che possono cambiare a seconda dell'ente di emissione dell'identità digitale stessa}
}
\newglossaryentry{designPatterns}{
name=Design Patterns,
description={Legare l'identità digitale all'individuo che la reclama attraverso vari meccanismi di riconoscimento che possono cambiare a seconda dell'ente di emissione dell'identità digitale stessa}
}
\newglossaryentry{poc}{
	name=Proof of Concept,
	description={si può tradurre in italiano con prova di concetto, si intende un'incompleta realizzazione o abbozzo di un certo progetto o metodo, con lo scopo di dimostrarne la fattibilità o la fondatezza di alcuni principi o concetti costituenti.}
}
\newglossaryentry{dlt}{
	name=Distributed Ledger,
	description={viene chiamato anche Registro Distribuito. Sono dei database distribuiti, ovvero di Ledgers (Libri Mastro) che possono essere aggiornati, gestiti, controllati e coordinati non più solo a livello centrale, ma in modo distribuito, da parte di tutti gli attori.}
}
\newglossaryentry{pow}{
	name=Proof of Work,
	description={metodologia utilizzata dalle \textit{Blockchain} per impedire attacchi di tipo Denial of Service o spam nella rete. Questo metodo di prevenzione richiede la risoluzione di un complesso gioco matematico la cui risoluzione fa si che il blocco sia inserito nella \textit{Blockchain}\cite{proof_of_work}}
}
\newglossaryentry{dos}{
	name=Denial of Service,
	description={attacco informatico in cui si fanno esaurire deliberatamente le risorse di un sistema informatico che fornisce un servizio ai client, fino a renderlo non più in grado di erogare il servizio richiesto\cite{Denial_of_service}}
}

 % database di termini
\makeglossaries


%**************************************************************
% Impostazioni di graphicx
%**************************************************************
\graphicspath{{immagini/}} % cartella dove sono riposte le immagini


%**************************************************************
% Impostazioni di hyperref
%**************************************************************
\hypersetup{
    %hyperfootnotes=false,
    %pdfpagelabels,
    %draft,	% = elimina tutti i link (utile per stampe in bianco e nero)
    colorlinks=true,
    linktocpage=true,
    pdfstartpage=1,
    pdfstartview=FitV,
    % decommenta la riga seguente per avere link in nero (per esempio per la stampa in bianco e nero)
    %colorlinks=false, linktocpage=false, pdfborder={0 0 0}, pdfstartpage=1, pdfstartview=FitV,
    breaklinks=true,
    pdfpagemode=UseNone,
    pageanchor=true,
    pdfpagemode=UseOutlines,
    plainpages=false,
    bookmarksnumbered,
    bookmarksopen=true,
    bookmarksopenlevel=1,
    hypertexnames=true,
    pdfhighlight=/O,
    %nesting=true,
    %frenchlinks,
    urlcolor=webbrown,
    linkcolor=RoyalBlue,
    citecolor=webgreen,
    %pagecolor=RoyalBlue,
    %urlcolor=Black, linkcolor=Black, citecolor=Black, %pagecolor=Black,
    pdftitle={\myTitle},
    pdfauthor={\textcopyright\ \myName, \myUni, \myFaculty},
    pdfsubject={},
    pdfkeywords={},
    pdfcreator={pdfLaTeX},
    pdfproducer={LaTeX}
}

%**************************************************************
% Impostazioni di itemize
%**************************************************************
\renewcommand{\labelitemi}{$\ast$}

%\renewcommand{\labelitemi}{$\bullet$}
%\renewcommand{\labelitemii}{$\cdot$}
%\renewcommand{\labelitemiii}{$\diamond$}
%\renewcommand{\labelitemiv}{$\ast$}


%**************************************************************
% Impostazioni di listings
%**************************************************************
\lstset{
    language=[LaTeX]Tex,%C++,
    keywordstyle=\color{RoyalBlue}, %\bfseries,
    basicstyle=\small\ttfamily,
    %identifierstyle=\color{NavyBlue},
    commentstyle=\color{Green}\ttfamily,
    stringstyle=\rmfamily,
    numbers=none, %left,%
    numberstyle=\scriptsize, %\tiny
    stepnumber=5,
    numbersep=8pt,
    showstringspaces=false,
    breaklines=true,
    frameround=ftff,
    frame=single
} 


%**************************************************************
% Impostazioni di xcolor
%**************************************************************
\definecolor{webgreen}{rgb}{0,.5,0}
\definecolor{webbrown}{rgb}{.6,0,0}


%**************************************************************
% Altro
%**************************************************************

\newcommand{\omissis}{[\dots\negthinspace]} % produce [...]

% eccezioni all'algoritmo di sillabazione
\hyphenation
{
    ma-cro-istru-zio-ne
    gi-ral-din
}

\newcommand{\sectionname}{sezione}
\addto\captionsitalian{\renewcommand{\figurename}{Figura}
                       \renewcommand{\tablename}{Tabella}}

\newcommand{\glsfirstoccur}{\ap{{[g]}}}

\newcommand{\intro}[1]{\emph{\textsf{#1}}}

%**************************************************************
% Environment per ``rischi''
%**************************************************************
\newcounter{riskcounter}                % define a counter
\setcounter{riskcounter}{0}             % set the counter to some initial value

%%%% Parameters
% #1: Title
\newenvironment{risk}[1]{
    \refstepcounter{riskcounter}        % increment counter
    \par \noindent                      % start new paragraph
    \textbf{\arabic{riskcounter}. #1}   % display the title before the 
                                        % content of the environment is displayed 
}{
    \par\medskip
}

\newcommand{\riskname}{Rischio}

\newcommand{\riskdescription}[1]{\textbf{\\Descrizione:} #1.}

\newcommand{\risksolution}[1]{\textbf{\\Soluzione:} #1.}

%**************************************************************
% Environment per ``use case''
%**************************************************************
\newcounter{usecasecounter}             % define a counter
\setcounter{usecasecounter}{0}          % set the counter to some initial value

%%%% Parameters
% #1: ID
% #2: Nome
\newenvironment{usecase}[2]{
    \renewcommand{\theusecasecounter}{\usecasename #1}  % this is where the display of 
                                                        % the counter is overwritten/modified
    \refstepcounter{usecasecounter}             % increment counter
    \vspace{10pt}
    \par \noindent                              % start new paragraph
    {\large \textbf{\usecasename #1: #2}}       % display the title before the 
                                                % content of the environment is displayed 
    \medskip
}{
    \medskip
}

\newcommand{\usecasename}{UC}

\newcommand{\usecaseactors}[1]{\textbf{\\Attori Principali:} #1. \vspace{4pt}}
\newcommand{\usecasepre}[1]{\textbf{\\Precondizioni:} #1. \vspace{4pt}}
\newcommand{\usecasedesc}[1]{\textbf{\\Descrizione:} #1. \vspace{4pt}}
\newcommand{\usecasepost}[1]{\textbf{\\Postcondizioni:} #1. \vspace{4pt}}
\newcommand{\usecasealt}[1]{\textbf{\\Scenario Alternativo:} #1. \vspace{4pt}}

%**************************************************************
% Environment per ``namespace description''
%**************************************************************

\newenvironment{namespacedesc}{
    \vspace{10pt}
    \par \noindent                              % start new paragraph
    \begin{description} 
}{
    \end{description}
    \medskip
}

\newcommand{\classdesc}[2]{\item[\textbf{#1:}] #2}