% !TEX encoding = UTF-8
% !TEX TS-program = pdflatex
% !TEX root = ../tesi.tex

%**************************************************************
\chapter{Codifica}
\label{cap:codifica}
%**************************************************************
Dopo aver analizato l'architettura del modulo \gls{ITF} insieme ai design pattern necessari alla realizzazione del prodotto passiamo ad analizzare nel dettaglio la struttura delle singole componenti.\\
I seguenti paragrafi analizzano quali metodi, attributi ed eventuali sottoclassi compongono i package più importanti necessari per la realizzazione del sistema.\\
Andremo ad analizzare:
\begin{itemzie}
	\item \textbf{Data\_Identity\_Wallet} contenuto nel package \textit{Identity Wallet Handler};
	\item \textbf{Trusted Third Party} contenuto nel package \textit{Trusted Third Party Handler};
	\item \textbf{Data\_ID\_List} contenuto nel package \textit{ID List};
	\item \textbf{Data\_Identity} contenuto nel package \textit{Identity};
	\item \textbf{Personally Identifiable Information};
	\item \textbf{Key\_List} contenuto nel package \textit{TTP Key List};
	\item \textbf{Data\_Service\_Provider}contenuto nel package \textit{Service Provider Handler}.
\end{itemzie}

\section{Identity Wallet Handler}
\subsection{Data\_Identity\_Wallet}
\section{Trusted Third Party Handler}
\subsection{Trusted Third Party}
\section{ID List}
\subsection{Data\_ID\_List}
\section{Identity}
\subsection{Data\_Identity}
\section{Personally Identifiable Information}
\section{TTP Key List}
\subsection{Key\_List}
\section{Service Provider Handler}
\subsection{Data\_Service\_Provider}
