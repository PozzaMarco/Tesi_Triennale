% !TEX encoding = UTF-8
% !TEX TS-program = pdflatex
% !TEX root = ../tesi.tex

%**************************************************************
\chapter{Valutazione finale}
\label{cap:valutazione_finale}
%**************************************************************
In questo capitolo viene fatto un \textit{excursus} riguardo gli obiettivi prefissati, il consuntivo orario finale in relazione al piano di lavoro concordato, le nuove conoscenze acquisite e, infine, una valutazione personale dello stage.
%**************************************************************
\section{Raggiungimento degli obiettivi}
Durante l'attività di stage qualche requisito riguardante le funzionalità ha subito, fermi restando gli obiettivi prefissati, dei leggeri cambiamenti in seguito a più approfonditi studi di dominio e della tecnologia \textit{Blockchain} scelta. Questo, fortunatamente, non ha avuto alcuna ripercussione sia dal punto di vista dei tempi stimanti che del raggiungimento degli obiettivi posti nel Piano di Lavoro.
\begin{longtable}{|r l|p{3cm}|p{8cm}|p{2cm}|}
	\hline
	\multicolumn{2}{|c|}{\textbf{ID}} & \textbf{IMPORTANZA} & \textbf{DESCRIZIONE} & \textbf{STATO}\tabularnewline
	\hline
	& O1 & Obbligatorio & Studio del dominio applicativo & Completato\\\hline
	& O2 & Obbligatorio & Studio delle possibili tecnologie \textit{Blockchain} da adottare & Completato \\\hline
	& O3 & Obbligatorio & Analisi comparativa delle tecnologie \textit{Blockchain} individuate & Completato \\\hline
	& O4 & Obbligatorio & Definizione dei vari casi d'uso per le funzionalità richieste (sopracitate) & Completato \\\hline
	& O4.1 & Obbligatorio & Definizione dei casi d'uso per la funzionalità di registrazione & Completato\\\hline
	& O4.2 & Obbligatorio & Definizione dei casi d'uso per a funzionalità di certificazione & Completato\\\hline
	& O4.3 & Obbligatorio & Definizione dei casi d'uso per la funzionalità di verifica & Completato\\\hline
	& O5 & Obbligatorio & Stesura del documento di analisi dei requisiti & Completato\\\hline
	& O6 & Obbligatorio & Stesura del documento contenete le specifiche architetturali del sistema & Completato\\\hline
	& O7 & Obbligatorio & Realizzazione del modulo \gls{ITF} & Completato\\\hline
	& O7.1 & Obbligatorio & Realizzazione della struttura e della gerarchia di classi necessaria alla realizzazione del modulo & Completato \\\hline
	& O7.2 & Obbligatorio & Implementazione dei metodi necessari per il corretto funzionamento del modulo & Completato\\\hline
	& D1 & Desiderabile & Produzione della documentazione relativa alle analisi svolte & Completato \\\hline
	& D2 & Desiderabile & Definizione dei casi d'uso per la funzionalità di presentazione dei dati & Completato \\\hline
	& D3 & Desiderabile & Definizione dei casi d'uso per la funzionalità di accesso al servizio & Completato\\\hline
	& D4 & Desiderabile & Stesura del documento di progettazione & Completato\\\hline
	& D5 & Desiderabile & Realizzazione dei test necessari alla verifica del corretto funzionamento del sistema & Completato\\\hline
	& D5.1 & Desiderabile & Implementazione dei test di unità per i singoli metodi & Completato\\\hline
	& D5.2 & Desiderabile & Implementazione dei test di integrazione per testare le classi & Completato\\\hline
	& D5.3 & Desiderabile & Implementazione dei test di integrazione per testare l'interazione tra tutte le classi che compongono il sistema & Completato\\\hline
	& F1 & Facoltativo & Stesura del documento relativo ai test di unità ed integrazione & Completato\\\hline	
	& F2 & Facoltativo & Stesura del documento relativo alla validazione del modulo (anomalie e bug) & Non completato\\\hline
	\caption{Tabella degli obiettivi raggiunti a fine stage}
\end{longtable}
Gli obiettivi la cui importanza era fondamentale per la realizzazione del modulo \gls{ITF} sono stati soddisfatti al 100\%.\\
Per quanto riguarda quelli desiderabili e facoltativi sono stati completati tutti tranne uno ovvero la\textit{ stesura del documento relativo alla validazione del modulo}. Questo perché durante gli ultimi giorni di stage, in accordo con il tutor aziendale, si è preferito dare più importanza al \textit{testing} di tutto il modulo completo comprensivo sia della mia parte che quella dell'altro stagista Simone Ballarin.
%**************************************************************
\section{Consuntivo finale}
Rispetto al piano di lavoro di inizio stage concordato con il tutor aziendale Sara Meneghetti e il responsabile di progetto Roberto Griggio, non ci sono state variazioni ai giorni di lavoro pianificate.\\
In conclusione, l'esperienza di stage ha avuto una durata complessiva di \textbf{320 ore} lavorative, come visibile in tabella, in cui viene riportata la fase lavorativa, già descritta nel paragrafo \hyperref[sec:pianificazione_del_lavoro]{pianificazione del lavoro} del \hyperref[cap:tecnologie_e_strumenti]{Capitolo 2}, le ore preventivate e le ore effettivamente svolte. 
\begin{longtable}{|r l|p{5cm}|p{4cm}|}
	\hline
	\multicolumn{2}{|c|}{\textbf{ATTIVITÀ}} & \textbf{ORE PREVENTIVATE} & \textbf{ORE EFFETTIVE}\tabularnewline
	\hline
	& Fase 1 & \centerline{40} & \centerline{40} \\\hline	
	& Fase 2 & \centerline{40} & \centerline{40}\\\hline
	& Fase 3 & \centerline{40} & \centerline{40}\\\hline
	& Fase 4 & \centerline{60} & \centerline{60}\\\hline
	& Fase 5 & \centerline{120} & \centerline{120}\\\hline
	& Fase 6 & \centerline{20} & \centerline{20}\\\hline	
	\caption{Riepilogo delle ore svolte durante lo stage}
\end{longtable}
%**************************************************************
\section{Conoscenze acquisite}
L'esperienza di stage vissuta mi ha permesso di ampliare le mie conoscenze sopratutto dal punto di vista tecnologico e organizzativo.\\
Ho imparato ad utilizzare moltissimi strumenti nuovi sopratutto per quanto riguarda il vasto mondo che circonda la tecnologia \textit{Blockchain} e le \gls{dapp} approfondendo ancora di più quanto già visto durante il progetto di Ingegneria del Software. Il dominio tecnologico che ho visto, analizzato, studiato e, in parte, utilizzato non era, quindi, una novità ma vedere le milioni di possibilità che offre è stato davvero entusiasmante e mi ha dato la voglia di approfondirlo ancora di più.
%**************************************************************
\section{Valutazione personale}
Personalmente ritengo questo stage un'esperienza fondamentale per la crescita professionale degli studenti.\\
Mi ha permesso di crescere molto sia a livello lavorativo che dei rapporti umani calati all'interno di un contesto totalmente diverso da quello che uno studente è, normalmente, abituato a vedere.\\
L'opportunità di stage che viene offerta dal nostro corso di studi è importantissima per la formazione che, un futuro neolaureato, deve avere prima di addentrarsi nel mondo lavorativo. Questo perché ti da l'opportunità di acquisire capacità e attitudini che non vengono insegnate durante il corso di studi come il rispettare gli orari lavorativi imposti o la propensione positiva verso sfide nuove che, inizialmente, sono difficili e, apparentemente, fuori dalla portata dello studente.\\
Le conoscenze acquisite durante il corso di Laurea in Informatica sono state importantissime per la buona riuscita dello stage, non tanto dal punto di vista delle tecnologie studiate, ma dalla mentalità e capacità di ragionamento e adattamento che il corso di studi porta a maturare.\\
Bisogna dire che l'ambiente lavorativo è di fondamentale importanza per la buona riuscita dello stage e Athesys S.r.l è un luogo che mette subito a proprio agio gli studenti che si apprestano a questa nuova esperienza. Questo perchè è un ambiente giovane e rilassato in cui il personale è molto competente e non mette pressioni di fronte a difficoltà o ritardi anzi offre supporto e sprona a fare il proprio meglio.\\
Le 8 settimane di stage sono state, quindi, un esperienza altamente positiva che mi ha permesso di crescere molto anche dal punto di vista professionale grazie alla piena responsabilità, autonomia e fiducia che il responsabile, Roberto Griggio, mi ha dato per la realizzazione del progetto.