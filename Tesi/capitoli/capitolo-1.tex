% !TEX encoding = UTF-8
% !TEX TS-program = pdflatex
% !TEX root = ../tesi.tex

%**************************************************************
\chapter{Introduzione}
\label{cap:introduzione}
%**************************************************************

Introduzione al contesto applicativo.\\

\noindent Esempio di utilizzo di un termine nel glossario \\
\gls{api}. \\

\noindent Esempio di citazione in linea \\
\cite{site:agile-manifesto}. \\

\noindent Esempio di citazione nel pie' di pagina \\
citazione\footcite{womak:lean-thinking} \\

\noindent Esempio di parola da glossario\\
\emph{parola}\glsfirstoccur;

In questo capitolo viene presentata, brevemente, l'azienda e la sua metodologia di lavoro che utilizza per lo sviluppo dei suoi progetti.\\
Verrà successivamente illustrato, in modo dettagliato, il progetto affrontato durante il periodo di stage partendo da una visione generale del problema ed arrivando alle possibili soluzioni studiate.

%**************************************************************
\section{L'azienda}
\subsection{Profilo aziendale}
Athesys nasce nel 2010 con sede a Padova dall’unione di professionisti IT che vantano una lunga esperienza in diversi ambiti tecnologici con l’obiettivo di mettere a fattor comune la loro competenza e tradurla nella capacità di erogare consulenza ad elevato contenuto tecnologico e progettuale a supporto delle complesse scelte strategiche che le aziende sono chiamate a prendere con sempre maggiore rapidità[\ref{1}].\\
I settori in cui l'azienda si muove sono diversi:
\begin{itemize}
	\item \textit{Identity and Access Management} - ...;
	\item hhh
	\item \textit{Database Management} - ;
	\item \textit{Business Intelligence} - .
\end{itemize}

\subsection{Metodologia di lavoro}

%**************************************************************
\section{L'idea}

Introduzione all'idea dello stage.

%**************************************************************
