% !TEX encoding = UTF-8
% !TEX TS-program = pdflatex
% !TEX root = ../tesi.tex

%**************************************************************
\chapter{Analisi dei requisiti}
\label{cap:analisi_dei_requisiti}
%**************************************************************
In questo capitolo verranno esposti i casi d'uso e i requisiti del modulo \gls{ITF} che erano oggetto del lavoro di stage.\\
Si procede con la descrizione dello stato del sistema, gli attori che vi partecipano, le pre-condizioni, le post-condizioni e gli scenari.\\
I casi d'uso principali sono associati ad un diagramma \gls{uml} 2.0 che riporta lo stesso codice identificativo e titolo del caso d'uso al quale si riferisce.

%**************************************************************
\section{Dominio}
\subsection{Caratteristiche degli attori}
L'interazione con il modulo \gls{ITF} coinvolge essenzialmente tre tipologie diverse di attori:
\begin{itemize}
	\item L'identity wallet;
	\item Un ente certificatore;
	\item Un service provider.
\end{itemize}

\subsubsection{Identity Wallet}
Questa componente del sistema ha il compito di generare, mantenere e presentare l'identità digitale dell'utente. A sua volta, l'\textit{Identity Wallet} interagisce con il sistema \gls{ITF} nel momento in cui un utente esegue una richiesta di accesso verso un \textit{service provider} il quale richiede l'autenticazione e la conferma delle credenziali insieme alla verifica di attributi specifici necessari per erogare i suoi servizi.\\
Il sistema \gls{monokee}, una volta che l'utente richiede l'accesso ad un \textit{service provider}, reindirizza la richiesta di accesso all'\gls{ITF}.
\subsubsection{Ente Certificatore (Trusted Third Party)}
La seconda tipologia di attori che interagiscono con il sistema da realizzare sono i \textit{Trusted Third Party}.\\
Questi, chiamati enti certificatori, hanno il compito di certificare le informazioni che sono presenti all'interno dell'\gls{ITF} in modo che, ogni qual volta che vengono richieste informazioni per poter accedere ad un \textit{service provider}, le informazioni che il sistema fornisce per l'autenticazione e/o l'erogazione di servizi, siano verificate da un ente esterno che garantisce la loro veridicità.
\subsubsection{Service Provider}
La terza e ultima tipologia di attori che partecipano al sistema sono i \textit{service provider}.
Questi interagiscono con l'\gls{ITF} durante la fase di verifica delle informazioni. Questi devono autenticare gli utenti tramite le loro credenziali e autorizzare l'erogazione di servizi dopo la verifica degli attributi
%**************************************************************
\subsection{Overview del sistema}
Una rappresentazione delle funzionalità che deve implementare il modulo \gls{ITF} può essere ripresa in figura \ref{fig:moduloITF}.\\\\
Il prodotto finale prevede l'interazione delle tre tipologie di attori allo scopo di garantire:
\begin{itemize}
	\item L'accesso sicuro al \textit{service provider} desiderato;
	\item L'erogazione dei servizi offerti dal \textit{service provider} in modo controllato.
\end{itemize}
Questo viene garantito dall'\gls{ITF} e dalla tecnologia \textit{Blockchain} che ne è alla base.\\
Di seguito viene descritto il flusso di chiamate che intercorrono affinchè un utente possa accedere ad uno o più servizi di sua scelta.\\\\
Quando un utente richiede dei servizi, da parte di un \textit{service provider}, il sistema viene interrogato e inizia lo scambio di informazioni che coinvolge l'\textit{identity wallet}, gli enti certifictori ed il \textit{service provider}.\\
Ogni qual volta un utente richiede l'accesso a dei servizi di un determinato \textit{service provider}, \gls{monokee} reindirizza la richiesta d'accesso all'\textit{identity wallet} che, a sua volta, invia la richiesta di accesso all'\gls{ITF} passandogli un identificativo dell'utente che ha fatto la richiesta di accesso insieme alle informazioni che il \textit{service provider} necessita per poter permettere l'erogazione dei suoi servizi.\\
Nell'\gls{ITF} viene eseguita una ricerca nella rete \textit{Blockchain} sottostante in modo da verificare se le informazioni richieste dal \textit{service provider} siano efffettivamente associate all'utente. Se il riscontro è positivo, il \textit{service provider} permette l'accesso all'utente altrimenti gli viene negato.\\\\
La seconda tipologia di attori che partecipano al sistema, gli enti certificatori, entrano in gioco durante la fase di registrazione di un nuovo utente o quando un amministratore di dominio associa degli attributi alle identità digitali dei suoi utenti.
Gli enti certificatori hanno il compito di verificare e certificare che, le asserzioni o gli attributi, di una particolare identità, siano conformi alle informazioni associate a quell'utente.\\
Questo ruolo è di fondamentale importanza all'interno del sistema che si sta sviluppando in quanto, essendo l'\gls{ITF} basata su una \textit{Blockchain}, tutte le informazioni che popolano la rete sono immutabili e, quindi, prima di essere definitivamente inserite in un blocco, devono essere verificate e approvate, dagli enti certificatori, tramite gli algoritmi di consenso.\\\\
La terza tipologia di attori, i \textit{service providers}, interagiscono con il sistema durante la verifica delle informazioni dell'identità dell'utente.\\
Questi, dopo che l'\textit{identity wallet} gli ha passato i record contenenti la chiave pubblica e il riferimento alla locazione del record cifrato nell'\gls{ITF}, iniziano il  controllo delle informazioni.\\
Il controllo viene fatto andando prima a calcolare l'\textit{hash crittografico} della chiave pubblica pervenuta dall'\textit{identity wallet} e poi andando a confrontare il risultato ottenuto con quello presente nell'\gls{ITF} recuperato grazie al riferimento alla sua locazione nella rete Blockchain.
In questa fase, il \textit{service provider} verifica i record contenente l'identità e qualsiasi altra credenziale per autenticare l'utente e verifica anche gli attributi necessari per autorizzare l'utente ad utilizzare i suoi servizi.
\section{Casi d'uso}
\subsection{Classificazione dei casi d'uso}
Ogni caso d'uso è classificato secondo la seguente convienzione:\\\\
\centerline{UC[codicePadre]\_[codiceFiglio]}\\\\
In cui i due codici rappresentano:
\begin{itemize}
	\item \textbf{codicePadre} - identifica univocamente il caso d'uso;
	\item \textbf{codiceFiglio} - identifica univocamente i sotto casi d'uso apparteneti ad un determinato "codicePadre".
\end{itemize}

\subsection{Descrizione dei casi d'uso}
Ogni caso d'uso viene descritto dalla seguente struttura:
\begin{itemize}
	\item \textbf{Descrizione} - Breve descrizione del caso d'uso che si sta modellando;
	\item \textbf{Attori Principali} - Indica l'attore principale del caso d'uso. In tutto il contesto dell'applicazione, gli attori saranno classificati come:
	\begin{itemize}
		\item Identity Wallet;
		\item Trusted Third Party (ente certificatore);
		\item Service Provider.
	\end{itemize}
	\item \textbf{Attori Secondari} - Indica l'attore che aiuta l'attore principale a realizzare quanto descritto dal caso d'uso;
	\item \textbf{Pre-Condizioni} - Specifica la condizione del sistema prima del verificarsi degli eventi descritti dal caso d'uso;
	\item \textbf{Post-Condizioni} - Specifica le condizioni del sistema dopo il verificarsi degli eventi descritti dal caso d'uso;
	\item \textbf{Scenario Principale} - Rappresenta il flusso principale degli eventi ovvero il caso in cui tutto funzioni come deve;
	\item \textbf{Estensioni} - Rappresenta il flusso secondario degli eventi nel caso in cui si verifichino degli errori nel flusso principale.
\end{itemize}

