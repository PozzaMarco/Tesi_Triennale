% !TeX spellcheck = it_IT
% !TEX encoding = UTF-8
% !TEX TS-program = pdflatex
% !TEX root = ../tesi.tex

%**************************************************************
% Organizzazione del Testo
%**************************************************************
\cleardoublepage
\phantomsection
\pdfbookmark{Organizzazione del Testo}{Organizzazione del Testo}
\begingroup
\let\clearpage\relax
\let\cleardoublepage\relax
\let\cleardoublepage\relax

\chapter*{Organizzazione del Testo}

\begin{description}
	\item[{\hyperref[cap:introduzione]{Capitolo 1}}] descrive l'azienda e le sue metodologie di lavoro. Inoltre viene descritto, nel dettaglio, il progetto di stage partendo da una panoramica generale del problema fino ad arrivare alla descrizione delle sue possibili soluzioni. Vengono esposti, infine, gli obiettivi da raggiungere e la pianificazione iniziale.
	
	\item[{\hyperref[cap:tecnologie_e_strumenti]{Capitolo 2}}] fornisce una descrizione delle tecnologie individuate e, dopo un dettagliato confronto, fornisce una panoramica di tutti gli strumenti utilizzati per l'implementazione,la deploy ed il testing del modulo ITF. 
	
	\item[{\hyperref[cap:analisi_dei_requisiti]{Capitolo 3}}] formalizza tutti i casi d'uso e i requisiti raccolti in fase di analisi dei requisiti.
	
	\item[{\hyperref[cap:progettazione]{Capitolo 4}}] illustra in modo esaustivo il Design Pattern architetturale e le scelte implementative adottate per la realizzazione del sistema.
	
	\item[{\hyperref[cap:codifica]{Capitolo 5}}] fornisce una visione degli aspetti più importanti inerenti la codifica ed il modulo realizzato andando a spiegare, nel dettaglio, le classi più importanti insieme ai campi dati ed i metodi che contengono.
	
	\item[{\hyperref[cap:verifica_validazione]{Capitolo 6}}] illustra le tecniche adottate per andare a verificare e validare il sistema finito in modo da assicurare il suo corretto funzionamento.
	
	\item[{\hyperref[cap:valutazione_finale]{Capitolo 7}}] fornisce una valutazione finale dello stage comprensiva degli obiettivi raggiunti, le conoscenze acquisite e le difficoltà riscontrate durante lo stage.
	
\end{description}

Riguardo la stesura del testo, relativamente al documento sono state adottate le seguenti convenzioni tipografiche:
\begin{itemize}
	\item gli acronimi, le abbreviazioni e i termini ambigui o di uso non comune menzionati vengono definiti nel glossario, situato alla fine del presente documento;
	\item per la prima occorrenza dei termini riportati nel glossario viene utilizzata la seguente nomenclatura: \emph{parola}\glsfirstoccur;
	\item i termini in lingua straniera o facenti parti del gergo tecnico sono evidenziati con il carattere \emph{corsivo}.
\end{itemize}

\endgroup			

\vfill