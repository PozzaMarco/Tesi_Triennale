% !TEX encoding = UTF-8
% !TEX TS-program = pdflatex
% !TEX root = ../tesi.tex

%**************************************************************
% Sommario
%**************************************************************
\cleardoublepage
\phantomsection
\pdfbookmark{Sommario}{Sommario}
\begingroup
\let\clearpage\relax
\let\cleardoublepage\relax
\let\cleardoublepage\relax

\chapter*{Sommario}

Il presente documento descrive il lavoro svolto duarante il periodo di stage, della durata di circa trecentoventi ore, del lauraendo Marco Pozza presso l'azienda Athesys S.r.l di Padova.\\\\
Lo scopo principale dello stage consistenva nella realizzazione di un layer di sicurezza, denominato Identity Trust Fabric (ITF) per integrare una tecnologia blockchain nei processi di accesso ai servizi e condivisione degli attributi di profilo.\\\\
Inoltre era richiesto che tale modulo venisse integrato come estensione delle funzionalità di un prodotto di Identity and Access Management già utilizzato dall'azienda.\\\\
Gli obbiettivi da raggiungere erano vari.\\
In primo luogo era richiesto lo studio di fattibilità, seguito da analisi dei requisiti e progettazione architetturale e di dettaglio dell'intero sistema.\\\\
In secondo luogo era richiesta l'implementazione del modulo ITF utilizzando gli strumenti e seguendo le specifiche individuate nel primo periodo di stage.\\\\
Terzo ed ultimo obbiettivo era la codifica di test per accertare che il modulo codificato funzionasse correttamente. Queste prove sono state, inoltre, riportate su un apposito documento.

%\vfill
%
%\selectlanguage{english}
%\pdfbookmark{Abstract}{Abstract}
%\chapter*{Abstract}
%
%\selectlanguage{italian}

\endgroup			

\vfill

