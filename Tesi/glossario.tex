
%**************************************************************
% Acronimi
%**************************************************************
\renewcommand{\acronymname}{Acronimi e abbreviazioni}
%\newacronym[description={\glslink{ITF}{Identity Trust Fabric}}]{ITF}{Identity Trust Fabric}
\newacronym{ITF}{ITF}{Identity Trust Fabric}
\newacronym{PII}{PII}{Personally Identifiable Information}
\newacronym{TTP}{TTP}{Trusted Third Party}

%**************************************************************
% Glossario
%**************************************************************
%\renewcommand{\glossaryname}{Glossario}
%\newglossaryentry{umlg}
%{
%	name=\glslink{uml}{UML},
%	text=UML,
%	sort=uml,
%	description={in ingegneria del software \emph{UML, Unified Modeling Language} (ing. linguaggio di modellazione unificato) è un linguaggio di modellazione e specifica basato sul paradigma object-oriented. L'\emph{UML} svolge un'importantissima funzione di ``lingua franca'' nella comunità della progettazione e programmazione a oggetti. Gran parte della letteratura di settore usa tale linguaggio per descrivere soluzioni analitiche e progettuali in modo sintetico e comprensibile a un vasto pubblico}
%}
\newglossaryentry{agile}
{
	name=Agile,
	description={
		 In ingegneria del software, è un insieme di metodi di sviluppo del software emersi a partire dai primi anni 2000 e fondati su un insieme di principi comuni, direttamente o indirettamente derivati dai principi del \textit{Manifesto per lo sviluppo agile del software}\cite{manifestoAgile}.\\
		 Tale manifesto si può riassumere in quattro punti:
		 \begin{enumerate}
		 	\item le persone e le interazioni sono più importanti dei processi e degli strumenti;
		 	\item è più importante avere software funzionante che documentazione
		 	\item bisogna collaborare con i clienti oltre che rispettare il contratto;
		 	\item bisogna essere pronti a rispodere ai cambiamenti oltre che aderire alla pianificazione.
		 \end{enumerate}
		 Un esempio di metodo di sviluppo di tipo agile è il metodo \gls{Scrum} \cite{scrum}}
}
\newglossaryentry{DataManagement}
{
	name=Data Management,
	description={Insieme di più attività che hanno lo scopo di sviluppare, eseguire e supervisionare progetti, politiche e programmi che controllano, proteggono, trasportano e aumentano il valore di dati e di risorse informative}
}
\newglossaryentry{DataMining}
{
  name=Data Mining,
  description={Insieme di tecniche e metodologie che hanno per oggetto l'estrazione di informazioni da grandi quantità di dati attraverso
  emtodi automatici o semi-automatici}
}
\newglossaryentry{Scrum}
{
	name=Scrum,
	description={Metodo di sviluppo software che rientra fra i metodi agile. Prevede di dividere il progetto in blocchi rapidi di lavoro (\textit{sprint}) alla fine di ciascuno dei quali crea un incremento del software. Esso indica come definire i dettagli del lavoro da fare nell'immediato futuro e prevede vari meeting con caratteristiche precise per creare occasioni di ispezione e controllo del lavoro svolto \cite{scrum}
	}
}
\newglossaryentry{ProductBacklog}
{
	name= Product Backlog,
	description={Insieme di requisiti per il lavoro da compiere, con priorità assegnate in base al valore di business}
}
\newglossaryentry{SprintBacklog}
{
	name= Sprint Backlog,
	description={Insieme degli elementi selezionati dal Product Backlog per lo \textit{sprint}}
}
\newglossaryentry{IAM}
{
	name=Identity and Access Management,
	description={L'Identity and Access Management è un sistema che si occupa della gestione delle informazioni riguardanti l'identità degli utenti e controlla l'accesso alle risorse informatiche. Nello specifico:
	\begin{itemize}
		\item \textbf{Identity Management} - censisce le utenze;
		\item \textbf{Access Management} - gestisce l'autenticazione e l'autorizzazione delle utenze. 
	\end{itemize}
	Con una soluzione di questo tipo, è possibile gestire centralmente gli utenti, le credenziali di sicurezza come chiavi di accesso e le autorizzazioni}
}
\newglossaryentry{identityWallet}{
	name=Identity Wallet,
	description={Portafogli digitali che contengono all'loro interno tutte le informazioni necessarie ad un utente per poter permettere l'autenticazione preciso servizio}
}
\newglossaryentry{monokee}{
name = Monokee,
description={Un \emph{\gls{IaaS}}\glsfirstoccur ideato e sviluppato da Athesys S.r.l per realizzare l'\gls{IAM}}
}
\newglossaryentry{IaaS}{
	name =Identity-as-a-Service,
	description ={È un servizio cloud che fornisce funzionalità di Identity and Access Management ad un sistema nel cloud e/o locale}
}
\newglossaryentry{identityProofing}{
	name=Identity Proofing,
	description={Conosciuta anche come \textit{Identity Verification} significa verificare ed autenticare l'identità di un utente prima che questo possa avere accesso ad uno o più servizi}
}
\newglossaryentry{identityBinding}{
	name=Identity Binding,
	description={Legare l'identità digitale all'individuo che la reclama attraverso vari meccanismi di riconoscimento che possono cambiare a seconda dell'ente di emissione dell'identità digitale stessa}
}
\newglossaryentry{designPatterns}{
name=Design Patterns,
description={Legare l'identità digitale all'individuo che la reclama attraverso vari meccanismi di riconoscimento che possono cambiare a seconda dell'ente di emissione dell'identità digitale stessa}
}
\newglossaryentry{poc}{
	name=Proof of Concept,
	description={Si può tradurre in italiano con "prova di concetto", si intende un'incompleta realizzazione o abbozzo di un certo progetto o metodo, con lo scopo di dimostrarne la fattibilità o la fondatezza di alcuni principi o concetti costituenti}
}
\newglossaryentry{dlt}{
	name=Distributed Ledger,
	description={Viene chiamato anche Registro Distribuito. Sono dei database distribuiti, ovvero di Ledgers (Libri Mastro) che possono essere aggiornati, gestiti, controllati e coordinati non più solo a livello centrale, ma in modo distribuito, da parte di tutti gli attori}
}
\newglossaryentry{pow}{
	name=Proof of Work,
	description={Metodologia utilizzata dalle \textit{Blockchain} per impedire attacchi di tipo Denial of Service o spam nella rete. Questo metodo di prevenzione richiede la risoluzione di un complesso gioco matematico la cui risoluzione fa si che il blocco sia inserito nella \textit{Blockchain} \cite{proof_of_work}}
}
\newglossaryentry{dos}{
	name=Denial of Service,
	description={Attacco informatico in cui si fanno esaurire deliberatamente le risorse di un sistema informatico che fornisce un servizio ai client, fino a renderlo non più in grado di erogare il servizio richiesto\cite{Denial_of_service}}
}
\newglossaryentry{sha256}{
	name=SHA 256,
	description={Famiglia di diverse funzioni crittografiche di \textit{hash} sviluppate a partire dal 1993.\\
	Come ongi algoritmo di \textit{hash}, \textit{SHA} produce un output di lunghezza fissa partendo da un input di lunghezza variabile. La sicurezza di questo algoritmo sta nel fatto che la funzione non è reversibile e che non è possibile creare intenzionalmente due messaggi diversi ma con lo stesso codice \textit{hash} \cite{sha_256}}
}
\newglossaryentry{gartner}{
	name=Gartner,
	description={Società multinazionale leader mondiale nella consulenza strategica, ricerca e analisi nel campo dell'Information Technogy}
}
\newglossaryentry{hyperledgerProject}{
	name=Hyperledger Project,
	description={Hyperledger è un progetto open source collaborativo della Linux Foundation creato e ospitato nella loro piattaforma dal 2015.
		Mira a promuovere le tecnologie Blockchain per le industrie assicurando responsabilità, fiducia e trasparenza tra i business partner. Come risultato, Hyperledger, rende il network aziendale e le transazioni molto più efficienti.
		Hyperledger offre varie piattaforme Blockchain}
}
\newglossaryentry{git}{
	name=Git,
	description={Software di controllo versione distribuito utilizzabile da interfaccia a riga di comando \cite{gitSite}}
}
\newglossaryentry{repository}{
	name=Repository,
	description={Ambiente di un sistema informativo in cui vengono gestiti i dati e le informazioni in formato digitale, archiviati sulla base di metadati che permettono una rapida individuazione, anche grazie alla creazione di tabelle relazionali \cite{gitSite}}
}
\newglossaryentry{metadati}{
	name=Metadati,
	description={Con il termine metadati (o anche \textit{meta tag}) si definiscono le informazioni che, associate ad una pagina web o ad una porzione di essa, ne descrivono il contenuto specificandone il contesto di riferimento. I metadati sono parte integrante del web semantico, termine coniato da Tim Berners-Lee e con il quale si fa riferimento alla possibilità di classificare i dati pubblicati sul web in maniera strutturata, in modo che la ricerca delle informazioni da parte degli utenti sia più efficace \cite{gitGuida}}
}
\newglossaryentry{commit}{
	name= Commit,
	description= {In ambiente Git, con commit si intende il comando che viene composto per validare le modifiche fatte sul proprio lavoro. Tramite commit il file si trova nell'HEAD, ma non ancora nel repository remoto \cite{gitSite} \cite{gitGuida}}
}
\newglossaryentry{event-driven}{
	name= Event-driven,
	description= {Il modello event-driven, o programmazione ad eventi, si basa su un concetto piuttosto semplice: si lancia una azione quando accade qualcosa. Ogni azione quindi risulta asincrona a differenza dei pattern di programmazione più comune in cui un'azione succede ad un'altra solo dopo che essa è stata completata. Ciò garantisce una certa efficienza delle applicazioni grazie ad un sistema di callback gestito a basso livello}
}
\newglossaryentry{opensource}{
	name= Open-source,
	description= {In Informatica, il termine inglese "open source" viene utilizzato per riferirsi ad un software di cui gli autori (più precisamente, i detentori dei diritti) rendono pubblico il codice sorgente, favorendone il libero studio e permettendo a programmatori indipendenti di apportarvi modifiche ed estensioni \cite{opensource}}
}
\newglossaryentry{uml}{
	name= UML,
	description={In Ingegneria del Software, UML, Unified Modeling Language, è un linguaggio di modellazione e specifica basato sul paradigma object-oriented. UML svolge un'importantissima funzione di "lingua franca" nella comunità della progettazione e programmazione a oggetti. Gran parte della letteratura di settore usa tale linguaggio per descrivere soluzioni analitiche e progettuali in modo sintetico e comprensibile a un vasto pubblico \cite{uml}}
}
\newglossaryentry{framework}{
	name= Framework,
	description={Architettura logica di supporto (spesso un'implementazione logica di un particolare design pattern) su cui un software può essere progettato e realizzato, spesso facilitandone lo sviluppo da parte del programmatore}
}
\newglossaryentry{dapp}{
	name= DApp,
	description={Abbreviazione per \textit{decentralized application}. Una Dapp è semplicemente del codice che gira su una rete decentralizzata peer-to-peer e si differenzia delle normali applicazioni in quanto girano su server centralizzati.\\
	Questo tipo di applicazioni può essere dotata di interfacce grafiche scritte in un qualsiasi linguaggio}
}
\newglossaryentry{mining}{
	name= Mining,
	description={Il mining è il metodo utilizzato dalle criptovalute per emettere moneta.\\
		La rete \textit{Blockchain} memorizza le transazioni all'interno di strutture di dati (blocchi). Affinchè un blocco possa essere aggiunto alla \textit{Blockchain} è necessario che un elaboratore lo "chiuda" trovando un particolare codice, che può essere unicamente calcolato. Questa operazione cristallizza il blocco, impedendo qualsiasi modifica futura, e chi trova tale codice è ricompensato con una certa quantità di criptomoneta, più tutte le tasse delle transazioni da lui inserite nel blocco, come incentivo alla "donazione" di tempo macchina \cite{mining}}
}
\newglossaryentry{evm}{
name= Ethereum Virtual Machine,
description={Macchina virtuale che permette l'uso degli script (smart contracts) scritti in \textit{Solidity} all'interno della \textit{Blockchain} Ethereum.\\
	È quindi una macchina virtuale, ossia un software in grado di emulare in tutto e per tutto una macchina fisica, attraverso un processo di virtualizzazione in cui vengono assegnate le risorse fisiche alle applicazioni che vi vengono eseguite}
}
\newglossaryentry{contract_oriented}{
	name= Contract-oriented,
	description={Sono una famiglia di linguaggi di programmazione tipicamente associati alle \gls{dapp} che sfruttano tecnologie \textit{Blockchain}.\\
	Si tratta di linguaggi di programmazione del tutto simili ai classici linguaggi "object-oriented" con la differenza che le classi, che formano gli oggetti nei linguaggi di programmazione classici sono chiamate contratti o "contracts" \cite{linuxFoundation}}
}
\newglossaryentry{stub}{
	name= Stub,
	description={Porzione di codice utilizzata in sostituzione di altre funzionalità software in quanto può simulare il comportamento di codice esistente o può simulare delle funzionalità non ancora sviluppate}
}
\newglossaryentry{topDown}{
	name= Top-down,
	description={In informatica, il top-down è una metodologia di progettazione che consiste nel formulare un programma procedendo per passi, tramite raffinamenti successivi, ponendo l’attenzione prima sui punti fondamentali e poi sui sottoproblemi più semplici \cite{topdown}}
}
\newglossaryentry{gas}{
	name= Gas,
	description={Nome associato ad una particolare unità di misura utilizzata nella \textit{Blockchain Ethereum} che misura quanto lavoro un azione o una serie di azioni necessitano per poter essere eseguite. Ogni operazione costa una certa quantità di \textit{gas} che deve essere pagata in \gls{ether} o suoi sottomultipli}
}
\newglossaryentry{ether}{
	name= Ether,
	description={Criptovaluta utilizzata dalla \textit{Blockchain Ethereum} come pagamento per l'esecuzione di transazioni}
}