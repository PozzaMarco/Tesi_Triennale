
%**************************************************************
% Acronimi
%**************************************************************
\renewcommand{\acronymname}{Acronimi e abbreviazioni}
%\newacronym[description={\glslink{ITF}{Identity Trust Fabric}}]{ITF}{Identity Trust Fabric}
\newacronym{ITF}{ITF}{Identity Trust Fabric}

%**************************************************************
% Glossario
%**************************************************************
%\renewcommand{\glossaryname}{Glossario}
%\newglossaryentry{umlg}
%{
%	name=\glslink{uml}{UML},
%	text=UML,
%	sort=uml,
%	description={in ingegneria del software \emph{UML, Unified Modeling Language} (ing. linguaggio di modellazione unificato) è un linguaggio di modellazione e specifica basato sul paradigma object-oriented. L'\emph{UML} svolge un'importantissima funzione di ``lingua franca'' nella comunità della progettazione e programmazione a oggetti. Gran parte della letteratura di settore usa tale linguaggio per descrivere soluzioni analitiche e progettuali in modo sintetico e comprensibile a un vasto pubblico}
%}
\newglossaryentry{agile}
{
	name=Agile,
	description={
		 In ingegneria del software, è un insieme di metodi di sviluppo del software emersi a partire dai primi anni 2000 e fondati su un insieme di principi comuni, direttamente o indirettamente derivati dai principi del \textit{Manifesto per lo sviluppo agile del software}\cite{manifestoAgile}.\\
		 Tale manifesto si può riassumere in quattro punti:
		 \begin{enumerate}
		 	\item le persone e le interazioni sono più importanti dei processi e degli strumenti;
		 	\item è più importante avere software funzionante che documentazione
		 	\item bisogna collaborare con i clienti oltre che rispettare il contratto;
		 	\item bisogna essere pronti a rispodere ai cambiamenti oltre che aderire alla pianificazione.
		 \end{enumerate}
		 Un esempio di metodo di sviluppo di tipo agile è il metodo \gls{Scrum}}
}
\newglossaryentry{DataManagement}
{
	name=Data Management,
	description={insime di più attività che hanno lo scopo di sviluppare, eseguire e supervisionare progetti, politiche e programmi che controllano, proteggono, trasportano e aumentano il valore di dati e di risorse informative}
}
\newglossaryentry{DataMining}
{
  name=Data Mining,
  description={Insieme di tecniche e metodologie che hanno per oggetto l'estrazione di informazioni da grandi quantità di dati attraverso
  emtodi automatici o semi-automatici}
}
\newglossaryentry{Scrum}
{
	name=Scrum,
	description={Metodo di sviluppo software che rientra fra i metodi agile. Prevede di dividere il progetto in blocchi rapidi di lavoro (\textit{sprint}) alla fine di ciascuno dei quali crea un incremento del software. Esso indica come definire i dettagli del lavoro da fare nell'immediato futuro e prevede vari meeting con caratteristiche precise per creare occasioni di ispezione e controllo del lavoro svolto \cite{scrum}
	}
}
\newglossaryentry{ProductBacklog}
{
	name= Product Backlog,
	description={Insime di requisiti per il lavoro da compiere, con priorità assegnate in base al valore di business}
}
\newglossaryentry{SprintBacklog}
{
	name= Sprint Backlog,
	description={Insieme degli elementi selezionati dal Product Backlog per lo \textit{sprint}}
}
\newglossaryentry{IAM}
{
	name=Identity and Access Management,
	description={L'Identity and Access Management è un sistema che si occupa della gestione delle informazioni riguardanti l'identità degli utenti e controlla l'accesso alle risorse informatiche. Nello specifico:
	\begin{itemize}
		\item \textbf{Identity Management} - censisce le utenze;
		\item \textbf{Access Management} - gestisce l'autenticazione e l'autorizzazione delle utenze. 
	\end{itemize}}
}
\newglossaryentry{identityWallet}{
	name=Identity Wallet,
	description={Portafogli digitali che contengono all'loro interno tutte le informazioni necessarie ad un utente per poter permettere l'autenticazione preciso servizio}
}
\newglossaryentry{monokee}{
name = Monokee,
description={Un \emph{\gls{IaaS}}\glsfirstoccur ideato e sviluppato da Athesys S.r.l per realizzare l'\gls{IAM}}
}
\newglossaryentry{IaaS}{
	name =Identity-as-a-Service,
	description ={È un servizio cloud che fornisce funzionalità di Identity and Access Management ad un sistema nel cloud e/o locale}
}
\newglossaryentry{identityProofing}{
	name=Identity Proofing,
	description={Conosciuta anche come \textit{Identity Verification} significa verificare ed autenticare l'identità di un utente prima che questo possa avere accesso ad uno o più servizi}
}
\newglossaryentry{identityBinding}{
	name=Identity Binding,
	description={Legare l'identità digitale all'individuo che la reclama attraverso vari meccanismi di riconoscimento che possono cambiare a seconda dell'ente di emissione dell'identità digitale stessa}
}

