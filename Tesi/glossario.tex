
%**************************************************************
% Acronimi
%**************************************************************
\renewcommand{\acronymname}{Acronimi e abbreviazioni}

%**************************************************************
% Glossario
%**************************************************************
%\renewcommand{\glossaryname}{Glossario}
%\newglossaryentry{umlg}
%{
%	name=\glslink{uml}{UML},
%	text=UML,
%	sort=uml,
%	description={in ingegneria del software \emph{UML, Unified Modeling Language} (ing. linguaggio di modellazione unificato) è un linguaggio di modellazione e specifica basato sul paradigma object-oriented. L'\emph{UML} svolge un'importantissima funzione di ``lingua franca'' nella comunità della progettazione e programmazione a oggetti. Gran parte della letteratura di settore usa tale linguaggio per descrivere soluzioni analitiche e progettuali in modo sintetico e comprensibile a un vasto pubblico}
%}
\newglossaryentry{agile}
{
	name=Agile,
	description={Prova
	}
}
\newglossaryentry{DataManagement}
{
	name=Data Management,
	description={insime di più attività che hanno lo scopo di sviluppare, eseguire e supervisionare progetti, politiche e programmi che controllano, proteggono, trasportano e aumentano il valore di dati e di risorse informative}
}
\newglossaryentry{DataMining}
{
  name=Data Mining,
  description={Insieme di tecniche e metodologie che hanno per oggetto l'estrazione di informazioni da grandi quantità di dati attraverso
  emtodi automatici o semi-automatici}
}
\newglossaryentry{Scrum}
{
	name=Scrum,
	description={Metodo di sviluppo software che rientra fra i metodi agile. Prevede di dividere il progetto in blocchi rapidi di lavoro (\textit{sprint}) alla fine di ciascuno dei quali crea un incremento del software. Esso indica come definire i dettagli del lavoro da fare nell'immediato futuro e prevede vari meeting con caratteristiche precise per creare occasioni di ispezione e controllo del lavoro svolto \cite{scrum}
	}
}
\newglossaryentry{ProductBacklog}
{
	name= Product Backlog,
	description={Insime di requisiti per il lavoro da compiere, con priorità assegnate in base al valore di business}
}
\newglossaryentry{SprintBacklog}
{
	name= Sprint Backlog,
	description={Insieme degli elementi selezionati dal Product Backlog per lo \textit{sprint}}
}

